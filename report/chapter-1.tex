%---------------------------------------------------------------------------------
\chapter{Introduction}
\label{chap:introduction}
%---------------------------------------------------------------------------------

The need for a better method of interaction with computers has been a great motivation for researchers in the speech processing community around the world to develop robust automatic speech recognition (ASR) systems. One major step which affects directly the performance of these systems is the detection of speech from the audio stream.

Speech detection, or voice activity detection (VAD) is the problem of determining the existence of speech from an acoustic signal.

Depending on the environment, non-speech can be silence, noise, music, or a variety of other acoustical signals.

This chapter provides the relevant background and introduce the typical structure of a VAD system. TODO.

\section{Background}
VAD is a fundamental task in almost any speech processing system, such as speech coding, speaker diarization and speech recognition. In speech coding, VAD helps to avoid the unnecessary coding and transmission of non-speech fragments, thus save bandwidth and computation. In speaker diarization and speech recognition, a good VAD may increase the purity of speaker models and greatly improve the accuracy. TODO.

% motivations:
% The introduction of interview speech in recent NIST Speaker Recognition Evaluations (SREs) has necessitated the development of robust voice activity detectors (VADs) that can work under very low signal-to-noise ratio

\section{Problem formulation}
Most VAD algorithms in the literature can be divided into two main components:
\begin{itemize}
  \item \textbf{feature extraction} which extracts an $N$-length sequence of feature vectors $\vt{y}=\{\vt{y}_1,\dots,\vt{y}_N\}$, $\vt{y}_i \in \realnumber^P$ from the given audio signal. Typically, these feature vectors are used train a model $\mathcal{M}$ which captures the difference of acoustic characteristics of speech and non-speech classes.
%  \item \textbf{acoustic modelling} which derives a set of models $\mathcal{M} = \{m_1,\dots,m_M\}$ characterizing the acoustic events of interests. %Almost all working VAD algorithms work with two events: speech and non-speech. In some applications, it's might be beneficial to define more specific events such as: music, laughing, knocking, paper shuffling, etc.
%      The purpose of these models during classification is to give $p(\vt{y}_k|m_i)$, the likelihood that a given observation $\vt{y}_k$ was generated by the model $m_i$, where $i=1 \dots M$.
	\item \textbf{decision rule} which searches for the most likely sequence of detection symbols $\hat{\vt{o}}\in \{\mathcal{O}_1,\mathcal{O}_2\}^N$ given the observation $\vt{y}$ and the acoustic model $\mathcal{M}$.
%  \item \textbf{decision rule} which searches for the most likely sequence of models $\hat{m}(k) \in \mathcal{M}$ given the observation $\vt{y}$ and the acoustic models:
      \begin{equation}
        \hat{o}_k = \argmax{o_i \in \{\mathcal{O}_1,\mathcal{O}_2\}} p(\vt{y}_k|o_i)
      \end{equation}
\end{itemize}


\section{Motivation}
Current problems with VAD are: poor performance under low SNR condition (below -5dB), non-stationary noise such as babble noise, or music.

\section{Objectives}
%To study the various existing algorithms, and to derive a new better one? :)
\section{Contribution}
%Not yet :)

\section{Thesis organization}
The rest of this thesis is organized as follow.

\begin{description}
\item[Chapter 2] studies the many existing algorithms for VAD, including features and decision rules proposed in the literature, divided into different categories. Their advantages and disadvantages are discussed.

\item[Chapter 3] shows the various experiments to evaluate and compare different VAD techniques. The experiments are designed to evaluate VAD features and decision rules separately.

\item[Chapter 4] develops the new techniques proposed for VAD, which enhance the performance...

\item[Chapter 5] finally summarizes and concudes the thesis.
\end{description}

%--------------------------------------------------------------------------------- 
